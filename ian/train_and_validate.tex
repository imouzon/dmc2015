% For LaTeX-Box: root = 
%%%%%%%%%%%%%%%%%%%%%%%%%%%%%%%%%%%%%%%%%%%%%%%%%%%%%%%%%%%%%%%%%%%%%%%%%%%%%%%%
%  File Name:
%  Purpose:
%
%  Creation Date: 04-05-2015
%  Last Modified: Thu May  7 09:43:22 2015
%  Created By:
%%%%%%%%%%%%%%%%%%%%%%%%%%%%%%%%%%%%%%%%%%%%%%%%%%%%%%%%%%%%%%%%%%%%%%%%%%%%%%%%

% example yaml
% ---
% title: Homework 2
% titleshort: HW 2
% instructions: Submit all code; show all methods
% author: Ian Mouzon
% authorshort: Mouzon
% contact: imouzon@iastate.edu
% grouplong: Machine Learning
% groupshort: STAT 602
% leader: Dr. Stephen Vardeman
% leadershort: Vardeman
% semester: Spring 2014
% assignment: Problem 1 - 5
% duedate: Monday May 4, 2015
% output:
%   usefulR::hw_template:
% ---

\PassOptionsToPackage{xcolor}{usenames,dvipsnames,svgnames,table}
\documentclass[10pt]{report}
\usepackage[T1]{fontenc}
\usepackage{lmodern}
\usepackage{pdfcolmk}
\usepackage{multirow}
\usepackage{graphicx}
\usepackage{pifont}
\usepackage{amsmath,amsfonts,amsthm,amssymb}
\usepackage{setspace}
\usepackage{Tabbing}
\usepackage{etoolbox}
\usepackage{fancyhdr}
\usepackage{lastpage}
\usepackage{listings}
\usepackage{extramarks}
\usepackage{enumerate}
\usepackage{soul,color}
\usepackage{graphicx,float,wrapfig}
\usepackage{amsmath,amssymb,rotating}
\usepackage{epsfig}
\usepackage{color}
\usepackage{hyperref}
\usepackage{animate}
\usepackage{array}
\usepackage{graphics, color}
\usepackage{graphicx}
\usepackage{epsfig}
\usepackage{setspace}
\usepackage{verbatim}
\usepackage[margin=1.0in]{geometry}
\usepackage{tikz}
\usepackage{mdframed}
\usepackage{clrscode3e}
\usepackage{formalHW}
%\usepackage[Ian Mouzon,]{formatHW}
\usepackage{fancyquote}
\usepackage{fancyenvironments}
\usepackage{mymathmacros}
\usepackage{algorithm}
\usepackage[noend]{algpseudocode}
\usepackage{pgfplots}

%set up fancy page header
%%
\pagestyle{fancy}
   \lhead{\href{mailto:imouzon@iastate.edu}{\nolinkurl{imouzon@iastate.edu}}}
   \lhead{authorshort}
   \chead{\ (\ ): }  
   \rhead{\firstxmark}
   \lfoot{\lastxmark}
   \cfoot{}
   \rfoot{Page\ \thepage\ of\ \pageref{LastPage}}
   \renewcommand\headrulewidth{0.4pt}
   \renewcommand\footrulewidth{0.4pt}

% pandoc syntax highlighting
\usepackage{color}
\usepackage{fancyvrb}
\newcommand{\VerbBar}{|}
\newcommand{\VERB}{\Verb[commandchars=\\\{\}]}
\DefineVerbatimEnvironment{Highlighting}{Verbatim}{commandchars=\\\{\}}
% Add ',fontsize=\small' for more characters per line
\newenvironment{Shaded}{}{}
\newcommand{\KeywordTok}[1]{\textcolor[rgb]{0.00,0.44,0.13}{\textbf{{#1}}}}
\newcommand{\DataTypeTok}[1]{\textcolor[rgb]{0.56,0.13,0.00}{{#1}}}
\newcommand{\DecValTok}[1]{\textcolor[rgb]{0.25,0.63,0.44}{{#1}}}
\newcommand{\BaseNTok}[1]{\textcolor[rgb]{0.25,0.63,0.44}{{#1}}}
\newcommand{\FloatTok}[1]{\textcolor[rgb]{0.25,0.63,0.44}{{#1}}}
\newcommand{\CharTok}[1]{\textcolor[rgb]{0.25,0.44,0.63}{{#1}}}
\newcommand{\StringTok}[1]{\textcolor[rgb]{0.25,0.44,0.63}{{#1}}}
\newcommand{\CommentTok}[1]{\textcolor[rgb]{0.38,0.63,0.69}{\textit{{#1}}}}
\newcommand{\OtherTok}[1]{\textcolor[rgb]{0.00,0.44,0.13}{{#1}}}
\newcommand{\AlertTok}[1]{\textcolor[rgb]{1.00,0.00,0.00}{\textbf{{#1}}}}
\newcommand{\FunctionTok}[1]{\textcolor[rgb]{0.02,0.16,0.49}{{#1}}}
\newcommand{\RegionMarkerTok}[1]{{#1}}
\newcommand{\ErrorTok}[1]{\textcolor[rgb]{1.00,0.00,0.00}{\textbf{{#1}}}}
\newcommand{\NormalTok}[1]{{#1}}


\begin{document}

%make title:
\thispagestyle{empty}%
\begin{center}%
    \renewcommand{\arraystretch}{1.5}%
    \begin{tabular}{c}%
       \Large{: }\\
       \\
       , \\
    \end{tabular}
\end{center}

\begin{center}
 \renewcommand{\arraystretch}{1.5}
 \begin{tabular*}{0.65\textwidth}{r@{:\hspace{.3cm}}l}
    \hline
    Name& Ian Mouzon\\
    email& \href{mailto:imouzon@iastate.edu}{\nolinkurl{imouzon@iastate.edu}}\\
    Instructions& \\
    Assignment& \\
    Due Date&  \\
    \hline
 \end{tabular*}
\end{center}

I am using the following packages to create this feature matrix:

\begin{Shaded}
\begin{Highlighting}[]
\KeywordTok{library}\NormalTok{(ggplot2)}
\KeywordTok{library}\NormalTok{(lubridate)}
\KeywordTok{library}\NormalTok{(dplyr)}
\KeywordTok{library}\NormalTok{(reshape2)}
\KeywordTok{library}\NormalTok{(sqldf)}
\end{Highlighting}
\end{Shaded}

\section{Getting the data}\label{getting-the-data}

I read the raw and clean data into R using the following simple
commands:

\begin{Shaded}
\begin{Highlighting}[]
\CommentTok{# training set ('historical data')}
\NormalTok{d =}\StringTok{ }\KeywordTok{readRDS}\NormalTok{(}\StringTok{"~/dmc2015/data/clean_data/clean_simple.rds"}\NormalTok{)}
\end{Highlighting}
\end{Shaded}

\subsection{Adding Batch ID}\label{adding-batch-id}

The data set \verb!clean_simple! does not have batch information. I will
have to add it in: I read the batch ID file as follows: \%-- readBatch:
R code (Code in Document)

\begin{Shaded}
\begin{Highlighting}[]
\CommentTok{# batch features in ~/dmc2015/features/feature_files/batchInfo_test.csv,}
\CommentTok{# batchInfo_train.csv}
\NormalTok{bit =}\StringTok{ }\KeywordTok{readRDS}\NormalTok{(}\StringTok{"~/dmc2015/features/feature_files/batchInfo_train.rds"}\NormalTok{)}
\NormalTok{bic =}\StringTok{ }\KeywordTok{readRDS}\NormalTok{(}\StringTok{"~/dmc2015/features/feature_files/batchInfo_test.rds"}\NormalTok{)}
\NormalTok{bi =}\StringTok{ }\KeywordTok{rbind}\NormalTok{(bit, bic)}
\end{Highlighting}
\end{Shaded}

Fix the formatting:

\begin{Shaded}
\begin{Highlighting}[]
\NormalTok{bi$couponsReceivedTime =}\StringTok{ }\KeywordTok{period_to_seconds}\NormalTok{(bi$couponsReceivedTime)/}\DecValTok{3600}
\NormalTok{bi$orderTimeTime =}\StringTok{ }\KeywordTok{period_to_seconds}\NormalTok{(bi$orderTimeTime)/}\DecValTok{3600}
\NormalTok{bi =}\StringTok{ }\KeywordTok{list}\NormalTok{(}\DataTypeTok{train =} \NormalTok{bi[}\KeywordTok{which}\NormalTok{(bi$orderID <=}\StringTok{ }\DecValTok{6053}\NormalTok{), ], }\DataTypeTok{class =} \NormalTok{bi[}\KeywordTok{which}\NormalTok{(bi$orderID >}\StringTok{ }
\StringTok{    }\DecValTok{6053}\NormalTok{), ])}
\end{Highlighting}
\end{Shaded}

Add it to the clean data:

\begin{Shaded}
\begin{Highlighting}[]
\NormalTok{d$train =}\StringTok{ }\NormalTok{d$train %>%}\StringTok{ }\KeywordTok{left_join}\NormalTok{(bi$train, }\DataTypeTok{by =} \StringTok{"orderID"}\NormalTok{)}
\NormalTok{d$class =}\StringTok{ }\NormalTok{d$class %>%}\StringTok{ }\KeywordTok{left_join}\NormalTok{(bi$class, }\DataTypeTok{by =} \StringTok{"orderID"}\NormalTok{)}
\end{Highlighting}
\end{Shaded}

\section{Creating Historical, Validation, and Training
Sets}\label{creating-historical-validation-and-training-sets}

\subsection{Set 1: Random Sampling on Coupons
Sets}\label{set-1-random-sampling-on-coupons-sets}

\begin{Shaded}
\begin{Highlighting}[]
\NormalTok{sample.set =}\StringTok{ }\NormalTok{d$train[, }\KeywordTok{c}\NormalTok{(}\StringTok{"orderID"}\NormalTok{, }\StringTok{"couponID1"}\NormalTok{, }\StringTok{"couponID2"}\NormalTok{, }\StringTok{"couponID3"}\NormalTok{)] %>%}\StringTok{ }
\StringTok{    }\KeywordTok{gather}\NormalTok{(couponCol, couponID, -orderID) %>%}\StringTok{ }\KeywordTok{arrange}\NormalTok{(orderID) %>%}\StringTok{ }\KeywordTok{select}\NormalTok{(orderID, }
    \NormalTok{couponID) %>%}\StringTok{ }\KeywordTok{mutate}\NormalTok{(}\DataTypeTok{couponID =} \KeywordTok{factor}\NormalTok{(couponID))}
\end{Highlighting}
\end{Shaded}

\begin{verbatim}
## Warning: attributes are not identical across measure variables; they will
## be dropped
\end{verbatim}

I would like to use 1/3 of my data as historical information in this
case. This means that I would like to get about 33\% of each coupons
total number.

That is, for \(n_i\) coupons of type \(i\) we want \(\frac{1}{3} n_i\)
coupons of type \(i\) in our historical pool.

\begin{Shaded}
\begin{Highlighting}[]
\KeywordTok{set.seed}\NormalTok{(}\DecValTok{1999}\NormalTok{)}
\NormalTok{H1.orderIDs =}\StringTok{ }\NormalTok{d$train$orderID[}\KeywordTok{sample}\NormalTok{(}\DecValTok{1}\NormalTok{:}\KeywordTok{nrow}\NormalTok{(d$train))[}\DecValTok{1}\NormalTok{:}\KeywordTok{round}\NormalTok{(}\KeywordTok{nrow}\NormalTok{(d$train)/}\DecValTok{3}\NormalTok{)]]}

\NormalTok{H1 =}\StringTok{ }\NormalTok{d$train[}\KeywordTok{which}\NormalTok{(d$train$orderID %in%}\StringTok{ }\NormalTok{H1.orderIDs), ]}
\NormalTok{T1 =}\StringTok{ }\NormalTok{d$train[}\KeywordTok{which}\NormalTok{(!(d$train$orderID %in%}\StringTok{ }\NormalTok{H1.orderIDs)), ]}

\KeywordTok{set.seed}\NormalTok{(}\DecValTok{7}\NormalTok{)}
\NormalTok{V1.orderIDs =}\StringTok{ }\NormalTok{T1$orderID[}\KeywordTok{sample}\NormalTok{(}\DecValTok{1}\NormalTok{:}\KeywordTok{nrow}\NormalTok{(T1))[}\DecValTok{1}\NormalTok{:}\KeywordTok{round}\NormalTok{(}\KeywordTok{nrow}\NormalTok{(T1)/}\DecValTok{3}\NormalTok{)]]}
\NormalTok{V1 =}\StringTok{ }\NormalTok{T1[}\KeywordTok{which}\NormalTok{(T1$orderID %in%}\StringTok{ }\NormalTok{V1.orderIDs), ]}
\NormalTok{T1 =}\StringTok{ }\NormalTok{T1[}\KeywordTok{which}\NormalTok{(!(T1$orderID %in%}\StringTok{ }\NormalTok{V1.orderIDs)), ]}

\CommentTok{# feature matrix}
\NormalTok{F1 =}\StringTok{ }\KeywordTok{list}\NormalTok{(}\DataTypeTok{orderids =} \KeywordTok{list}\NormalTok{(}\DataTypeTok{h =} \NormalTok{H1.orderIDs, }\DataTypeTok{v =} \NormalTok{V1.orderIDs), }\DataTypeTok{H =} \NormalTok{H1, }\DataTypeTok{V =} \NormalTok{V1, }
    \DataTypeTok{T =} \NormalTok{T1, }\DataTypeTok{C =} \NormalTok{d$class)}
\end{Highlighting}
\end{Shaded}

\subsection{Set 2: Random Sampling on
Users}\label{set-2-random-sampling-on-users}

\begin{Shaded}
\begin{Highlighting}[]
\KeywordTok{set.seed}\NormalTok{(}\DecValTok{1979}\NormalTok{)}
\NormalTok{H2.userIDs =}\StringTok{ }\KeywordTok{sample}\NormalTok{(}\KeywordTok{unique}\NormalTok{(d$train$userID))[}\DecValTok{1}\NormalTok{:}\KeywordTok{round}\NormalTok{(}\KeywordTok{length}\NormalTok{(}\KeywordTok{unique}\NormalTok{(d$train$userID))/}\DecValTok{3}\NormalTok{)]}

\NormalTok{H2 =}\StringTok{ }\NormalTok{d$train[}\KeywordTok{which}\NormalTok{(d$train$userID %in%}\StringTok{ }\NormalTok{H2.userIDs), ]}
\NormalTok{T2 =}\StringTok{ }\NormalTok{d$train[}\KeywordTok{which}\NormalTok{(!(d$train$userID %in%}\StringTok{ }\NormalTok{H2.userIDs)), ]}

\KeywordTok{set.seed}\NormalTok{(}\DecValTok{17}\NormalTok{)}
\NormalTok{V2.userIDs =}\StringTok{ }\NormalTok{T2$userID[}\KeywordTok{sample}\NormalTok{(}\DecValTok{1}\NormalTok{:}\KeywordTok{nrow}\NormalTok{(T2))[}\DecValTok{1}\NormalTok{:}\KeywordTok{round}\NormalTok{(}\KeywordTok{nrow}\NormalTok{(T2)/}\DecValTok{3}\NormalTok{)]]}
\NormalTok{V2 =}\StringTok{ }\NormalTok{T2[}\KeywordTok{which}\NormalTok{(T2$userID %in%}\StringTok{ }\NormalTok{V2.userIDs), ]}
\NormalTok{T2 =}\StringTok{ }\NormalTok{T2[}\KeywordTok{which}\NormalTok{(!(T2$userID %in%}\StringTok{ }\NormalTok{V2.userIDs)), ]}

\CommentTok{# feature matrix}
\NormalTok{F2 =}\StringTok{ }\KeywordTok{list}\NormalTok{(}\DataTypeTok{userids =} \KeywordTok{list}\NormalTok{(}\DataTypeTok{h =} \NormalTok{H2.userIDs, }\DataTypeTok{v =} \NormalTok{V2.userIDs), }\DataTypeTok{H =} \NormalTok{H2, }\DataTypeTok{V =} \NormalTok{V2, }\DataTypeTok{T =} \NormalTok{T2, }
    \DataTypeTok{C =} \NormalTok{d$class)}
\end{Highlighting}
\end{Shaded}

\subsection{Set 3: Batch Number}\label{set-3-batch-number}

\begin{Shaded}
\begin{Highlighting}[]
\NormalTok{H3 =}\StringTok{ }\NormalTok{d$train[}\KeywordTok{which}\NormalTok{(d$train$batchID %in%}\StringTok{ }\DecValTok{1}\NormalTok{:}\DecValTok{7}\NormalTok{), ]}
\NormalTok{T3 =}\StringTok{ }\NormalTok{d$train[}\KeywordTok{which}\NormalTok{(d$train$batchID ==}\StringTok{ }\DecValTok{8}\NormalTok{), ]}
\NormalTok{V3 =}\StringTok{ }\NormalTok{d$train[}\KeywordTok{which}\NormalTok{(d$train$batchID ==}\StringTok{ }\DecValTok{9}\NormalTok{), ]}
\NormalTok{C3 =}\StringTok{ }\NormalTok{d$class}
\NormalTok{F3 =}\StringTok{ }\KeywordTok{list}\NormalTok{(}\DataTypeTok{H =} \NormalTok{H3, }\DataTypeTok{V =} \NormalTok{V3, }\DataTypeTok{T =} \NormalTok{T3, }\DataTypeTok{C =} \NormalTok{d$class)}
\end{Highlighting}
\end{Shaded}

\section{Write out the sets}\label{write-out-the-sets}

\begin{Shaded}
\begin{Highlighting}[]
\KeywordTok{saveRDS}\NormalTok{(F1, }\DataTypeTok{file =} \StringTok{"../data/featureMatrix/HTVset1.rds"}\NormalTok{)}
\KeywordTok{saveRDS}\NormalTok{(F2, }\DataTypeTok{file =} \StringTok{"../data/featureMatrix/HTVset2.rds"}\NormalTok{)}
\KeywordTok{saveRDS}\NormalTok{(F3, }\DataTypeTok{file =} \StringTok{"../data/featureMatrix/HTVset3.rds"}\NormalTok{)}
\end{Highlighting}
\end{Shaded}

\end{document}
