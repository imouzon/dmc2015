


\PassOptionsToPackage{xcolor}{usenames,dvipsnames,svgnames,table}
\documentclass[10pt]{report}
\usepackage[]{graphicx}\usepackage[]{xcolor}
%% maxwidth is the original width if it is less than linewidth
%% otherwise use linewidth (to make sure the graphics do not exceed the margin)
\makeatletter
\def\maxwidth{ %
  \ifdim\Gin@nat@width>\linewidth
    \linewidth
  \else
    \Gin@nat@width
  \fi
}
\makeatother

\definecolor{fgcolor}{rgb}{0.345, 0.345, 0.345}
\newcommand{\hlnum}[1]{\textcolor[rgb]{0.686,0.059,0.569}{#1}}%
\newcommand{\hlstr}[1]{\textcolor[rgb]{0.192,0.494,0.8}{#1}}%
\newcommand{\hlcom}[1]{\textcolor[rgb]{0.678,0.584,0.686}{\textit{#1}}}%
\newcommand{\hlopt}[1]{\textcolor[rgb]{0,0,0}{#1}}%
\newcommand{\hlstd}[1]{\textcolor[rgb]{0.345,0.345,0.345}{#1}}%
\newcommand{\hlkwa}[1]{\textcolor[rgb]{0.161,0.373,0.58}{\textbf{#1}}}%
\newcommand{\hlkwb}[1]{\textcolor[rgb]{0.69,0.353,0.396}{#1}}%
\newcommand{\hlkwc}[1]{\textcolor[rgb]{0.333,0.667,0.333}{#1}}%
\newcommand{\hlkwd}[1]{\textcolor[rgb]{0.737,0.353,0.396}{\textbf{#1}}}%

\usepackage{framed}
\makeatletter
\newenvironment{kframe}{%
 \def\at@end@of@kframe{}%
 \ifinner\ifhmode%
  \def\at@end@of@kframe{\end{minipage}}%
  \begin{minipage}{\columnwidth}%
 \fi\fi%
 \def\FrameCommand##1{\hskip\@totalleftmargin \hskip-\fboxsep
 \colorbox{shadecolor}{##1}\hskip-\fboxsep
     % There is no \\@totalrightmargin, so:
     \hskip-\linewidth \hskip-\@totalleftmargin \hskip\columnwidth}%
 \MakeFramed {\advance\hsize-\width
   \@totalleftmargin\z@ \linewidth\hsize
   \@setminipage}}%
 {\par\unskip\endMakeFramed%
 \at@end@of@kframe}
\makeatother

\definecolor{shadecolor}{rgb}{.97, .97, .97}
\definecolor{messagecolor}{rgb}{0, 0, 0}
\definecolor{warningcolor}{rgb}{1, 0, 1}
\definecolor{errorcolor}{rgb}{1, 0, 0}
\newenvironment{knitrout}{}{} % an empty environment to be redefined in TeX

\usepackage{alltt}
\newcommand{\SweaveOpts}[1]{}  % do not interfere with LaTeX
\newcommand{\SweaveInput}[1]{} % because they are not real TeX commands
\newcommand{\Sexpr}[1]{}       % will only be parsed by R


\usepackage[T1]{fontenc}
\usepackage{lmodern}
\usepackage{pdfcolmk}
\usepackage{multirow}
\usepackage{graphicx}
\usepackage{pifont}
\usepackage{amsmath,amsfonts,amsthm,amssymb}
\usepackage{setspace}
\usepackage{Tabbing}
\usepackage{etoolbox}
\usepackage{fancyhdr}
\usepackage{lastpage}
\usepackage{listings}
\usepackage{extramarks}
\usepackage{enumerate}
\usepackage{soul,color}
\usepackage{graphicx,float,wrapfig}
\usepackage{amsmath,amssymb,rotating}
\usepackage{epsfig}
\usepackage{color}
\usepackage{hyperref}
\usepackage{animate}
\usepackage{array}
\usepackage{graphics, color}
\usepackage{graphicx}
\usepackage{epsfig}
\usepackage{setspace}
\usepackage{verbatim}
\usepackage[margin=1.0in]{geometry}
\usepackage{tikz}
\usepackage{mdframed}
\usepackage{clrscode3e}
\usepackage{formalHW}
\usepackage[DMC,none]{formatHW}
\usepackage{fancyquote}
\usepackage{fancyenvironments}
\usepackage{mymathmacros}
\usepackage{algorithm}
\usepackage[noend]{algpseudocode}
\usepackage{pgfplots}





\begin{document}

% For LaTeX-Box: root = feature_files/R/featureMatrix_v0.1.tex 
%%%%%%%%%%%%%%%%%%%%%%%%%%%%%%%%%%%%%%%%%%%%%%%%%%%%%%%%%%%%%%%%%%%%%%%%%%%%%%%%
%  File Name:
%  Purpose:
%
%  Creation Date: 22-04-2015
%  Last Modified: Sat Apr 25 00:25:28 2015
%  Created By:
%%%%%%%%%%%%%%%%%%%%%%%%%%%%%%%%%%%%%%%%%%%%%%%%%%%%%%%%%%%%%%%%%%%%%%%%%%%%%%%%
%-- Set parent file


\HWinfo{April 25 2015}{}{}
\titleheader
I am using the following packages:
%-- paks: R code (Code in Document)
\begin{knitrout}
\definecolor{shadecolor}{rgb}{0.969, 0.969, 0.969}\color{fgcolor}\begin{kframe}
\begin{alltt}
\hlkwd{library}\hlstd{(ggplot2)}
\hlkwd{library}\hlstd{(lubridate)}
\hlkwd{library}\hlstd{(dplyr)}
\hlkwd{library}\hlstd{(reshape2)}
\hlkwd{library}\hlstd{(sqldf)}
\end{alltt}
\end{kframe}
\end{knitrout}
and my working directory is set to \verb!dmc2015/ian/features/feature_files/R/!.

\section{Reading the Data}
This file updates the feature matrix \verb!featureMatrix_v0!. 
I am adding features Pete created:
%-- readDat: R code (Code in Document)
\begin{knitrout}
\definecolor{shadecolor}{rgb}{0.969, 0.969, 0.969}\color{fgcolor}\begin{kframe}
\begin{alltt}
\hlcom{# read in the training set}
\hlstd{featureMatrix} \hlkwb{<-} \hlkwd{readRDS}\hlstd{(}\hlstr{"~/dmc2015/data/featureMatrix/featMat_v0.0.rds"}\hlstd{)}

\hlcom{# split the list feature matrix}
\hlstd{trn} \hlkwb{<-} \hlstd{featureMatrix}\hlopt{$}\hlstd{train}
\hlstd{cls} \hlkwb{<-} \hlstd{featureMatrix}\hlopt{$}\hlstd{class}

\hlcom{# get the raw training and test set too for}
\hlcom{# reference}
\hlstd{trn.raw} \hlkwb{<-} \hlkwd{read.delim}\hlstd{(}\hlstr{"~/dmc2015/data/raw_data/DMC_2015_orders_train.txt"}\hlstd{,}
    \hlkwc{stringsAsFactors} \hlstd{=} \hlnum{FALSE}\hlstd{,} \hlkwc{sep} \hlstd{=} \hlstr{"|"}\hlstd{,} \hlkwc{quote} \hlstd{=} \hlstr{""}\hlstd{)}
\hlstd{cls.raw} \hlkwb{<-} \hlkwd{read.delim}\hlstd{(}\hlstr{"~/dmc2015/data/raw_data/DMC_2015_orders_class.txt"}\hlstd{,}
    \hlkwc{stringsAsFactors} \hlstd{=} \hlnum{FALSE}\hlstd{,} \hlkwc{sep} \hlstd{=} \hlstr{"|"}\hlstd{,} \hlkwc{quote} \hlstd{=} \hlstr{""}\hlstd{)}
\end{alltt}
\end{kframe}
\end{knitrout}

\section{Reading the Features}
Pete stored his features in files that can be read into R using the following: 
%-- readBatch: R code (Code in Document)
\begin{knitrout}
\definecolor{shadecolor}{rgb}{0.969, 0.969, 0.969}\color{fgcolor}\begin{kframe}
\begin{alltt}
\hlcom{# Training and test features of coupon counts}
\hlstd{nCoupTrain} \hlkwb{<-} \hlkwd{read.csv}\hlstd{(}\hlstr{"../nCoupTrain.csv"}\hlstd{)}
\hlstd{nCoupClass} \hlkwb{<-} \hlkwd{read.csv}\hlstd{(}\hlstr{"../nCoupClass.csv"}\hlstd{)}

\hlcom{# NAs should be 0}
\hlkwa{for} \hlstd{(i} \hlkwa{in} \hlnum{1}\hlopt{:}\hlkwd{nrow}\hlstd{(nCoupTrain)) \{}
    \hlkwa{for} \hlstd{(j} \hlkwa{in} \hlnum{1}\hlopt{:}\hlkwd{ncol}\hlstd{(nCoupTrain)) \{}
        \hlkwa{if} \hlstd{(}\hlkwd{is.na}\hlstd{(nCoupTrain[i, j]))}
            \hlstd{nCoupTrain[i, j]} \hlkwb{<-} \hlnum{0}
    \hlstd{\}}
\hlstd{\}}

\hlcom{# NAs should be 0}
\hlkwa{for} \hlstd{(i} \hlkwa{in} \hlnum{1}\hlopt{:}\hlkwd{nrow}\hlstd{(nCoupClass)) \{}
    \hlkwa{for} \hlstd{(j} \hlkwa{in} \hlnum{1}\hlopt{:}\hlkwd{ncol}\hlstd{(nCoupClass)) \{}
        \hlkwa{if} \hlstd{(}\hlkwd{is.na}\hlstd{(nCoupClass[i, j]))}
            \hlstd{nCoupClass[i, j]} \hlkwb{<-} \hlnum{0}
    \hlstd{\}}
\hlstd{\}}
\end{alltt}
\end{kframe}
\end{knitrout}
Add the batch features:
%-- addFeatures: R code (Code in Document)
\begin{knitrout}
\definecolor{shadecolor}{rgb}{0.969, 0.969, 0.969}\color{fgcolor}\begin{kframe}
\begin{alltt}
\hlcom{# It's so easy with dplyr}
\hlstd{trn} \hlkwb{<-} \hlstd{trn} \hlopt \hlkwd{left_join}\hlstd{(nCoupTrain,} \hlkwc{by} \hlstd{=} \hlstr{"orderID"}\hlstd{)}
\hlstd{cls} \hlkwb{<-} \hlstd{cls} \hlopt \hlkwd{left_join}\hlstd{(nCoupClass,} \hlkwc{by} \hlstd{=} \hlstr{"orderID"}\hlstd{)}
\end{alltt}
\end{kframe}
\end{knitrout}
61 Features!

\section{Writing the Feature Matrix}
We can save the features as CSV files and R objects (using \verb!saveRDS! and writing the training and test sets as lists):
%-- : R code (Code in Document)
\begin{knitrout}
\definecolor{shadecolor}{rgb}{0.969, 0.969, 0.969}\color{fgcolor}\begin{kframe}
\begin{alltt}
\hlkwd{write.csv}\hlstd{(trn,} \hlkwc{file} \hlstd{=} \hlstr{"~/dmc2015/data/featureMatrix/train_ver1.0.csv"}\hlstd{,}
    \hlkwc{row.names} \hlstd{=} \hlnum{FALSE}\hlstd{,} \hlkwc{na} \hlstd{=} \hlstr{""}\hlstd{,} \hlkwc{quote} \hlstd{=} \hlnum{FALSE}\hlstd{)}
\hlkwd{write.csv}\hlstd{(cls,} \hlkwc{file} \hlstd{=} \hlstr{"~/dmc2015/data/featureMatrix/class_ver1.0.csv"}\hlstd{,}
    \hlkwc{row.names} \hlstd{=} \hlnum{FALSE}\hlstd{,} \hlkwc{na} \hlstd{=} \hlstr{""}\hlstd{,} \hlkwc{quote} \hlstd{=} \hlnum{FALSE}\hlstd{)}

\hlstd{featMat} \hlkwb{<-} \hlkwd{list}\hlstd{(}\hlkwc{train} \hlstd{= trn,} \hlkwc{class} \hlstd{= cls)}
\hlkwd{saveRDS}\hlstd{(featMat,} \hlkwc{file} \hlstd{=} \hlstr{"~/dmc2015/data/featureMatrix/featMat_v1.0.rds"}\hlstd{)}
\end{alltt}
\end{kframe}
\end{knitrout}
\end{document}
