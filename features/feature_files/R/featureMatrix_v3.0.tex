% For LaTeX-Box: root = 
%%%%%%%%%%%%%%%%%%%%%%%%%%%%%%%%%%%%%%%%%%%%%%%%%%%%%%%%%%%%%%%%%%%%%%%%%%%%%%%%
%  File Name:
%  Purpose:
%
%  Creation Date: 04-05-2015
%  Last Modified: Mon May  4 18:37:05 2015
%  Created By:
%%%%%%%%%%%%%%%%%%%%%%%%%%%%%%%%%%%%%%%%%%%%%%%%%%%%%%%%%%%%%%%%%%%%%%%%%%%%%%%%

% example yaml
% ---
% title: Homework 2
% titleshort: HW 2
% instructions: Submit all code; show all methods
% author: Ian Mouzon
% authorshort: Mouzon
% contact: imouzon@iastate.edu
% grouplong: Machine Learning
% groupshort: STAT 602
% leader: Dr. Stephen Vardeman
% leadershort: Vardeman
% semester: Spring 2014
% assignment: Problem 1 - 5
% duedate: Monday May 4, 2015
% output:
%   usefulR::hw_template:
% ---

\PassOptionsToPackage{xcolor}{usenames,dvipsnames,svgnames,table}
\documentclass[10pt]{report}
\usepackage[T1]{fontenc}
\usepackage{lmodern}
\usepackage{pdfcolmk}
\usepackage{multirow}
\usepackage{graphicx}
\usepackage{pifont}
\usepackage{amsmath,amsfonts,amsthm,amssymb}
\usepackage{setspace}
\usepackage{Tabbing}
\usepackage{etoolbox}
\usepackage{fancyhdr}
\usepackage{lastpage}
\usepackage{listings}
\usepackage{extramarks}
\usepackage{enumerate}
\usepackage{soul,color}
\usepackage{graphicx,float,wrapfig}
\usepackage{amsmath,amssymb,rotating}
\usepackage{epsfig}
\usepackage{color}
\usepackage{hyperref}
\usepackage{animate}
\usepackage{array}
\usepackage{graphics, color}
\usepackage{graphicx}
\usepackage{epsfig}
\usepackage{setspace}
\usepackage{verbatim}
\usepackage[margin=1.0in]{geometry}
\usepackage{tikz}
\usepackage{mdframed}
\usepackage{clrscode3e}
\usepackage{formalHW}
%\usepackage[Ian Mouzon,]{formatHW}
\usepackage{fancyquote}
\usepackage{fancyenvironments}
\usepackage{mymathmacros}
\usepackage{algorithm}
\usepackage[noend]{algpseudocode}
\usepackage{pgfplots}

%set up fancy page header
%%
\pagestyle{fancy}
   \lhead{\href{mailto:imouzon@iastate.edu}{\nolinkurl{imouzon@iastate.edu}}}
   \lhead{authorshort}
   \chead{DMC 2015\ (\href{mailto:DMC@ISU}{\nolinkurl{DMC@ISU}}\ Spring 2015): FeatMat3}  
   \rhead{\firstxmark}
   \lfoot{\lastxmark}
   \cfoot{}
   \rfoot{Page\ \thepage\ of\ \pageref{LastPage}}
   \renewcommand\headrulewidth{0.4pt}
   \renewcommand\footrulewidth{0.4pt}

% pandoc syntax highlighting
\usepackage{color}
\usepackage{fancyvrb}
\newcommand{\VerbBar}{|}
\newcommand{\VERB}{\Verb[commandchars=\\\{\}]}
\DefineVerbatimEnvironment{Highlighting}{Verbatim}{commandchars=\\\{\}}
% Add ',fontsize=\small' for more characters per line
\newenvironment{Shaded}{}{}
\newcommand{\KeywordTok}[1]{\textcolor[rgb]{0.00,0.44,0.13}{\textbf{{#1}}}}
\newcommand{\DataTypeTok}[1]{\textcolor[rgb]{0.56,0.13,0.00}{{#1}}}
\newcommand{\DecValTok}[1]{\textcolor[rgb]{0.25,0.63,0.44}{{#1}}}
\newcommand{\BaseNTok}[1]{\textcolor[rgb]{0.25,0.63,0.44}{{#1}}}
\newcommand{\FloatTok}[1]{\textcolor[rgb]{0.25,0.63,0.44}{{#1}}}
\newcommand{\CharTok}[1]{\textcolor[rgb]{0.25,0.44,0.63}{{#1}}}
\newcommand{\StringTok}[1]{\textcolor[rgb]{0.25,0.44,0.63}{{#1}}}
\newcommand{\CommentTok}[1]{\textcolor[rgb]{0.38,0.63,0.69}{\textit{{#1}}}}
\newcommand{\OtherTok}[1]{\textcolor[rgb]{0.00,0.44,0.13}{{#1}}}
\newcommand{\AlertTok}[1]{\textcolor[rgb]{1.00,0.00,0.00}{\textbf{{#1}}}}
\newcommand{\FunctionTok}[1]{\textcolor[rgb]{0.02,0.16,0.49}{{#1}}}
\newcommand{\RegionMarkerTok}[1]{{#1}}
\newcommand{\ErrorTok}[1]{\textcolor[rgb]{1.00,0.00,0.00}{\textbf{{#1}}}}
\newcommand{\NormalTok}[1]{{#1}}


\begin{document}

%make title:
\thispagestyle{empty}%
\begin{center}%
    \renewcommand{\arraystretch}{1.5}%
    \begin{tabular}{c}%
       \Large{DMC 2015: Data Mining Coupons}\\
       Feature Matrix 3\\
       Spring 2015, Iowa State University Data Mining Cup Team\\
    \end{tabular}
\end{center}

\begin{center}
 \renewcommand{\arraystretch}{1.5}
 \begin{tabular*}{0.65\textwidth}{r@{:\hspace{.3cm}}l}
    \hline
    Name& Ian Mouzon\\
    email& \href{mailto:imouzon@iastate.edu}{\nolinkurl{imouzon@iastate.edu}}\\
    Instructions& \\
    Assignment& \\
    Due Date&  May 5th, 2015\\
    \hline
 \end{tabular*}
\end{center}

I am using the following packages to create this feature matrix:

\begin{Shaded}
\begin{Highlighting}[]
\KeywordTok{library}\NormalTok{(ggplot2)}
\KeywordTok{library}\NormalTok{(lubridate)}
\KeywordTok{library}\NormalTok{(dplyr)}
\KeywordTok{library}\NormalTok{(reshape2)}
\KeywordTok{library}\NormalTok{(sqldf)}
\end{Highlighting}
\end{Shaded}

\section{Getting the data}\label{getting-the-data}

I read the raw and clean data into R using the following simple
commands:

\begin{Shaded}
\begin{Highlighting}[]
\CommentTok{# training set ('historical data')}
\NormalTok{trn.raw =}\StringTok{ }\KeywordTok{read.delim}\NormalTok{(}\StringTok{"~/dmc2015/data/raw_data/DMC_2015_orders_train.txt"}\NormalTok{, }\DataTypeTok{stringsAsFactors =} \OtherTok{FALSE}\NormalTok{, }
    \DataTypeTok{sep =} \StringTok{"|"}\NormalTok{, }\DataTypeTok{quote =} \StringTok{""}\NormalTok{)}
\NormalTok{trn =}\StringTok{ }\KeywordTok{read.csv}\NormalTok{(}\StringTok{"~/dmc2015/data/clean_data/train_simple_name.csv"}\NormalTok{, }\DataTypeTok{stringsAsFactors =} \OtherTok{FALSE}\NormalTok{, }
    \DataTypeTok{na.strings =} \StringTok{""}\NormalTok{)[, }\KeywordTok{names}\NormalTok{(trn.raw)]}

\CommentTok{# test set ('future data')}
\NormalTok{cls.raw =}\StringTok{ }\KeywordTok{read.delim}\NormalTok{(}\StringTok{"~/dmc2015/data/raw_data/DMC_2015_orders_class.txt"}\NormalTok{, }\DataTypeTok{stringsAsFactors =} \OtherTok{FALSE}\NormalTok{, }
    \DataTypeTok{sep =} \StringTok{"|"}\NormalTok{, }\DataTypeTok{quote =} \StringTok{""}\NormalTok{)}
\NormalTok{cls =}\StringTok{ }\KeywordTok{read.csv}\NormalTok{(}\StringTok{"~/dmc2015/data/clean_data/test_simple_name.csv"}\NormalTok{, }\DataTypeTok{stringsAsFactors =} \OtherTok{FALSE}\NormalTok{, }
    \DataTypeTok{na.strings =} \StringTok{""}\NormalTok{)[, }\KeywordTok{names}\NormalTok{(cls.raw)]}
\end{Highlighting}
\end{Shaded}

\subsection{Formatting the data}\label{formatting-the-data}

The data in thi raw form has the following formatting:

\begin{Shaded}
\begin{Highlighting}[]
\KeywordTok{str}\NormalTok{(trn)}
\end{Highlighting}
\end{Shaded}

\begin{verbatim}
## 'data.frame':    6053 obs. of  32 variables:
##  $ orderID        : int  1 2 3 4 5 6 7 8 9 10 ...
##  $ orderTime      : chr  "2015-01-06 09:38:35" "2015-01-06 10:03:19" "2015-01-06 10:08:13" "2015-01-06 13:23:23" ...
##  $ userID         : chr  "user1" "user2" "user3" "user4" ...
##  $ couponsReceived: chr  "2015-01-06 09:34:53" "2015-01-06 10:00:44" "2015-01-06 09:29:16" "2015-01-06 13:13:12" ...
##  $ couponID1      : chr  "cpn1" "cpn4" "cpn7" "cpn10" ...
##  $ price1         : num  3.24 2.32 7.92 2.5 12.27 ...
##  $ basePrice1     : num  5.4 1.59 2.64 2.08 2.45 1.25 2.59 2.04 4.54 1.25 ...
##  $ reward1        : num  1.57 1.57 1.26 1.57 1.26 0.63 1.26 0.94 1.26 0.94 ...
##  $ premiumProduct1: int  0 0 1 0 0 0 1 1 0 0 ...
##  $ brand1         : chr  "brand1" "brand2" "brand4" "brand6" ...
##  $ productGroup1  : chr  "prod1" "prod4" "prod7" "prod10" ...
##  $ categoryIDs1   : chr  "cat1:cat2" "cat5" "cat1:cat7" "cat1" ...
##  $ couponID2      : chr  "cpn2" "cpn5" "cpn8" "cpn11" ...
##  $ price2         : num  5.19 3.7 4.17 3.66 5.74 4.35 10 4.82 3.98 3.06 ...
##  $ basePrice2     : num  0.57 1.85 1.39 0.73 0.88 1.44 1.33 0.6 1.99 0.61 ...
##  $ reward2        : num  1.57 0.94 1.26 1.57 1.57 1.26 0.94 1.57 1.88 1.26 ...
##  $ premiumProduct2: int  0 0 1 1 1 0 0 0 0 1 ...
##  $ brand2         : chr  "brand2" "brand3" "brand4" "brand4" ...
##  $ productGroup2  : chr  "prod2" "prod5" "prod8" "prod11" ...
##  $ categoryIDs2   : chr  "cat3:cat4" "cat6" "cat3:cat7" "cat3:cat7:cat4" ...
##  $ couponID3      : chr  "cpn3" "cpn6" "cpn9" "cpn12" ...
##  $ price3         : num  12.92 3.89 2.73 5.74 4.91 ...
##  $ basePrice3     : num  12.92 0.06 0.88 4.25 2.45 ...
##  $ reward3        : num  2.2 2.2 1.26 1.57 1.57 3.14 1.88 1.57 1.26 1.26 ...
##  $ premiumProduct3: int  0 0 0 0 1 0 0 1 0 0 ...
##  $ brand3         : chr  "brand3" "brand2" "brand5" "brand5" ...
##  $ productGroup3  : chr  "prod3" "prod6" "prod9" "prod12" ...
##  $ categoryIDs3   : chr  "cat5:cat4:cat2" "cat6" "cat5:cat2" "cat5:cat2" ...
##  $ coupon1Used    : int  0 1 0 1 0 1 0 1 0 1 ...
##  $ coupon2Used    : int  1 0 0 1 0 0 0 1 0 0 ...
##  $ coupon3Used    : int  0 1 0 0 0 1 0 1 0 1 ...
##  $ basketValue    : num  188 186 208 186 272 ...
\end{verbatim}

We would like to properly format the data/time variables and convert
factors to factors:

\begin{Shaded}
\begin{Highlighting}[]
\CommentTok{# first we stack together}
\NormalTok{trn$dsn =}\StringTok{ "train"}
\NormalTok{cls$dsn =}\StringTok{ "class"}
\NormalTok{d =}\StringTok{ }\KeywordTok{rbind}\NormalTok{(trn, cls)}
\NormalTok{d$dsn =}\StringTok{ }\KeywordTok{factor}\NormalTok{(d$dsn, }\DataTypeTok{levels =} \KeywordTok{c}\NormalTok{(}\StringTok{"train"}\NormalTok{, }\StringTok{"class"}\NormalTok{))}

\CommentTok{# formatting orderTime}
\NormalTok{d$orderTime =}\StringTok{ }\KeywordTok{ymd_hms}\NormalTok{(d$orderTime)}

\CommentTok{# formatting couponsReceived}
\NormalTok{d$couponsReceived =}\StringTok{ }\KeywordTok{ymd_hms}\NormalTok{(d$couponsReceived)}

\CommentTok{# formatting userID}
\NormalTok{d$userID =}\StringTok{ }\KeywordTok{as.factor}\NormalTok{(d$userID)}

\CommentTok{# formatting brands:}
\NormalTok{d$brand1 =}\StringTok{ }\KeywordTok{factor}\NormalTok{(d$brand1, }\DataTypeTok{levels =} \KeywordTok{paste0}\NormalTok{(}\StringTok{"brand"}\NormalTok{, }\DecValTok{1}\NormalTok{:}\KeywordTok{length}\NormalTok{(}\KeywordTok{unique}\NormalTok{(}\KeywordTok{c}\NormalTok{(d$brand1, }
    \NormalTok{d$brand2, d$brand3)))))}
\NormalTok{d$brand2 =}\StringTok{ }\KeywordTok{factor}\NormalTok{(d$brand2, }\DataTypeTok{levels =} \KeywordTok{paste0}\NormalTok{(}\StringTok{"brand"}\NormalTok{, }\DecValTok{1}\NormalTok{:}\KeywordTok{length}\NormalTok{(}\KeywordTok{unique}\NormalTok{(}\KeywordTok{c}\NormalTok{(d$brand1, }
    \NormalTok{d$brand2, d$brand3)))))}
\NormalTok{d$brand3 =}\StringTok{ }\KeywordTok{factor}\NormalTok{(d$brand3, }\DataTypeTok{levels =} \KeywordTok{paste0}\NormalTok{(}\StringTok{"brand"}\NormalTok{, }\DecValTok{1}\NormalTok{:}\KeywordTok{length}\NormalTok{(}\KeywordTok{unique}\NormalTok{(}\KeywordTok{c}\NormalTok{(d$brand1, }
    \NormalTok{d$brand2, d$brand3)))))}

\CommentTok{# formatting productGroups:}
\NormalTok{d$productGroup1 =}\StringTok{ }\KeywordTok{factor}\NormalTok{(d$productGroup1, }\DataTypeTok{levels =} \KeywordTok{paste0}\NormalTok{(}\StringTok{"prod"}\NormalTok{, }\DecValTok{1}\NormalTok{:}\KeywordTok{length}\NormalTok{(}\KeywordTok{unique}\NormalTok{(}\KeywordTok{c}\NormalTok{(d$productGroup1, }
    \NormalTok{d$productGroup2, d$productGroup3)))))}
\NormalTok{d$productGroup2 =}\StringTok{ }\KeywordTok{factor}\NormalTok{(d$productGroup2, }\DataTypeTok{levels =} \KeywordTok{paste0}\NormalTok{(}\StringTok{"prod"}\NormalTok{, }\DecValTok{1}\NormalTok{:}\KeywordTok{length}\NormalTok{(}\KeywordTok{unique}\NormalTok{(}\KeywordTok{c}\NormalTok{(d$productGroup1, }
    \NormalTok{d$productGroup2, d$productGroup3)))))}
\NormalTok{d$productGroup3 =}\StringTok{ }\KeywordTok{factor}\NormalTok{(d$productGroup3, }\DataTypeTok{levels =} \KeywordTok{paste0}\NormalTok{(}\StringTok{"prod"}\NormalTok{, }\DecValTok{1}\NormalTok{:}\KeywordTok{length}\NormalTok{(}\KeywordTok{unique}\NormalTok{(}\KeywordTok{c}\NormalTok{(d$productGroup1, }
    \NormalTok{d$productGroup2, d$productGroup3)))))}

\CommentTok{# formatting couponIDs:}
\NormalTok{d$couponID1 =}\StringTok{ }\KeywordTok{factor}\NormalTok{(d$couponID1, }\DataTypeTok{levels =} \KeywordTok{paste0}\NormalTok{(}\StringTok{"cpn"}\NormalTok{, }\DecValTok{1}\NormalTok{:}\KeywordTok{length}\NormalTok{(}\KeywordTok{unique}\NormalTok{(}\KeywordTok{c}\NormalTok{(d$couponID1, }
    \NormalTok{d$couponID2, d$couponID3)))))}
\NormalTok{d$couponID2 =}\StringTok{ }\KeywordTok{factor}\NormalTok{(d$couponID2, }\DataTypeTok{levels =} \KeywordTok{paste0}\NormalTok{(}\StringTok{"cpn"}\NormalTok{, }\DecValTok{1}\NormalTok{:}\KeywordTok{length}\NormalTok{(}\KeywordTok{unique}\NormalTok{(}\KeywordTok{c}\NormalTok{(d$couponID1, }
    \NormalTok{d$couponID2, d$couponID3)))))}
\NormalTok{d$couponID3 =}\StringTok{ }\KeywordTok{factor}\NormalTok{(d$couponID3, }\DataTypeTok{levels =} \KeywordTok{paste0}\NormalTok{(}\StringTok{"cpn"}\NormalTok{, }\DecValTok{1}\NormalTok{:}\KeywordTok{length}\NormalTok{(}\KeywordTok{unique}\NormalTok{(}\KeywordTok{c}\NormalTok{(d$couponID1, }
    \NormalTok{d$couponID2, d$couponID3)))))}
\end{Highlighting}
\end{Shaded}

which gives the following structure:

\begin{Shaded}
\begin{Highlighting}[]
\KeywordTok{str}\NormalTok{(d)}
\end{Highlighting}
\end{Shaded}

\begin{verbatim}
## 'data.frame':    6722 obs. of  33 variables:
##  $ orderID        : int  1 2 3 4 5 6 7 8 9 10 ...
##  $ orderTime      : POSIXct, format: "2015-01-06 09:38:35" "2015-01-06 10:03:19" ...
##  $ userID         : Factor w/ 3096 levels "user1","user10",..: 1 1112 2223 2431 2542 2653 2764 2875 2986 2 ...
##  $ couponsReceived: POSIXct, format: "2015-01-06 09:34:53" "2015-01-06 10:00:44" ...
##  $ couponID1      : Factor w/ 3446 levels "cpn1","cpn2",..: 1 4 7 10 13 16 19 22 25 28 ...
##  $ price1         : num  3.24 2.32 7.92 2.5 12.27 ...
##  $ basePrice1     : num  5.4 1.59 2.64 2.08 2.45 1.25 2.59 2.04 4.54 1.25 ...
##  $ reward1        : num  1.57 1.57 1.26 1.57 1.26 0.63 1.26 0.94 1.26 0.94 ...
##  $ premiumProduct1: int  0 0 1 0 0 0 1 1 0 0 ...
##  $ brand1         : Factor w/ 28 levels "brand1","brand2",..: 1 2 4 6 6 2 4 4 3 2 ...
##  $ productGroup1  : Factor w/ 231 levels "prod1","prod2",..: 1 4 7 10 7 15 7 7 20 15 ...
##  $ categoryIDs1   : chr  "cat1:cat2" "cat5" "cat1:cat7" "cat1" ...
##  $ couponID2      : Factor w/ 4159 levels "cpn1","cpn2",..: 2 5 8 11 14 17 20 23 26 29 ...
##  $ price2         : num  5.19 3.7 4.17 3.66 5.74 4.35 10 4.82 3.98 3.06 ...
##  $ basePrice2     : num  0.57 1.85 1.39 0.73 0.88 1.44 1.33 0.6 1.99 0.61 ...
##  $ reward2        : num  1.57 0.94 1.26 1.57 1.57 1.26 0.94 1.57 1.88 1.26 ...
##  $ premiumProduct2: int  0 0 1 1 1 0 0 0 0 1 ...
##  $ brand2         : Factor w/ 54 levels "brand1","brand2",..: 2 3 4 4 4 3 3 3 2 4 ...
##  $ productGroup2  : Factor w/ 399 levels "prod1","prod2",..: 2 5 8 11 13 16 18 11 14 22 ...
##  $ categoryIDs2   : chr  "cat3:cat4" "cat6" "cat3:cat7" "cat3:cat7:cat4" ...
##  $ couponID3      : Factor w/ 4573 levels "cpn1","cpn2",..: 3 6 9 12 15 18 21 24 27 30 ...
##  $ price3         : num  12.92 3.89 2.73 5.74 4.91 ...
##  $ basePrice3     : num  12.92 0.06 0.88 4.25 2.45 ...
##  $ reward3        : num  2.2 2.2 1.26 1.57 1.57 3.14 1.88 1.57 1.26 1.26 ...
##  $ premiumProduct3: int  0 0 0 0 1 0 0 1 0 0 ...
##  $ brand3         : Factor w/ 52 levels "brand1","brand2",..: 3 2 5 5 4 3 2 4 3 3 ...
##  $ productGroup3  : Factor w/ 413 levels "prod1","prod2",..: 3 6 9 12 14 17 14 19 21 23 ...
##  $ categoryIDs3   : chr  "cat5:cat4:cat2" "cat6" "cat5:cat2" "cat5:cat2" ...
##  $ coupon1Used    : int  0 1 0 1 0 1 0 1 0 1 ...
##  $ coupon2Used    : int  1 0 0 1 0 0 0 1 0 0 ...
##  $ coupon3Used    : int  0 1 0 0 0 1 0 1 0 1 ...
##  $ basketValue    : num  188 186 208 186 272 ...
##  $ dsn            : Factor w/ 2 levels "train","class": 1 1 1 1 1 1 1 1 1 1 ...
\end{verbatim}

We can save this raw dataset first by splitting the pieces back up:

\begin{Shaded}
\begin{Highlighting}[]
\NormalTok{trn =}\StringTok{ }\NormalTok{d[}\KeywordTok{which}\NormalTok{(d$dsn ==}\StringTok{ "train"}\NormalTok{), ]}
\NormalTok{trn =}\StringTok{ }\NormalTok{trn[, -}\KeywordTok{which}\NormalTok{(}\KeywordTok{names}\NormalTok{(trn) ==}\StringTok{ "dsn"}\NormalTok{)]}
\NormalTok{cls =}\StringTok{ }\NormalTok{d[}\KeywordTok{which}\NormalTok{(d$dsn ==}\StringTok{ "class"}\NormalTok{), ]}
\NormalTok{cls =}\StringTok{ }\NormalTok{cls[, -}\KeywordTok{which}\NormalTok{(}\KeywordTok{names}\NormalTok{(cls) ==}\StringTok{ "dsn"}\NormalTok{)]}
\end{Highlighting}
\end{Shaded}

Then by writing the following \verb!RDS! file

\begin{Shaded}
\begin{Highlighting}[]
\KeywordTok{saveRDS}\NormalTok{(}\KeywordTok{list}\NormalTok{(}\DataTypeTok{train =} \NormalTok{trn, }\DataTypeTok{class =} \NormalTok{cls), }\DataTypeTok{file =} \StringTok{"~/dmc2015/data/clean_data/clean_simple.rds"}\NormalTok{)}
\end{Highlighting}
\end{Shaded}

\section{Adding the Features}\label{adding-the-features}

\subsection{Batch ID}\label{batch-id}

I read the batch ID file as follows: \%-- readBatch: R code (Code in
Document)

\begin{Shaded}
\begin{Highlighting}[]
\CommentTok{# batch features in ~/dmc2015/features/feature_files/batchInfo_test.csv,}
\CommentTok{# batchInfo_train.csv}
\NormalTok{bit =}\StringTok{ }\KeywordTok{readRDS}\NormalTok{(}\StringTok{"~/dmc2015/features/feature_files/batchInfo_train.rds"}\NormalTok{)}
\NormalTok{bic =}\StringTok{ }\KeywordTok{readRDS}\NormalTok{(}\StringTok{"~/dmc2015/features/feature_files/batchInfo_test.rds"}\NormalTok{)}
\NormalTok{bi =}\StringTok{ }\KeywordTok{rbind}\NormalTok{(bit, bic)}
\end{Highlighting}
\end{Shaded}

\subsubsection{Check the batch to make sure it's
clean}\label{check-the-batch-to-make-sure-its-clean}

Check for no missing features:

\begin{Shaded}
\begin{Highlighting}[]
\CommentTok{# check for no missing values}
\KeywordTok{nrow}\NormalTok{(bi) -}\StringTok{ }\KeywordTok{length}\NormalTok{(}\KeywordTok{complete.cases}\NormalTok{(bi))}
\end{Highlighting}
\end{Shaded}

\begin{verbatim}
## [1] 0
\end{verbatim}

Fix the formatting:

\begin{Shaded}
\begin{Highlighting}[]
\NormalTok{bi$couponsReceivedTime =}\StringTok{ }\KeywordTok{period_to_seconds}\NormalTok{(bi$couponsReceivedTime)/}\DecValTok{3600}
\NormalTok{bi$orderTimeTime =}\StringTok{ }\KeywordTok{period_to_seconds}\NormalTok{(bi$orderTimeTime)/}\DecValTok{3600}

\NormalTok{bi =}\StringTok{ }\KeywordTok{list}\NormalTok{(}\DataTypeTok{train =} \NormalTok{bi[}\KeywordTok{which}\NormalTok{(bi$orderID <=}\StringTok{ }\DecValTok{6053}\NormalTok{), ], }\DataTypeTok{class =} \NormalTok{bi[}\KeywordTok{which}\NormalTok{(bi$orderID >}\StringTok{ }
\StringTok{    }\DecValTok{6053}\NormalTok{), ])}
\end{Highlighting}
\end{Shaded}

\subsection{User Visit Features}\label{user-visit-features}

\begin{Shaded}
\begin{Highlighting}[]
\NormalTok{uv =}\StringTok{ }\KeywordTok{readRDS}\NormalTok{(}\StringTok{"~/dmc2015/features/feature_files/UserVisitFeatures.rds"}\NormalTok{)}

\CommentTok{# no missing values}
\KeywordTok{length}\NormalTok{(}\KeywordTok{complete.cases}\NormalTok{(uv$train)) -}\StringTok{ }\KeywordTok{nrow}\NormalTok{(uv$train)}
\end{Highlighting}
\end{Shaded}

\begin{verbatim}
## [1] 0
\end{verbatim}

\subsection{Coupon Basket Stats}\label{coupon-basket-stats}

\begin{Shaded}
\begin{Highlighting}[]
\NormalTok{cb =}\StringTok{ }\KeywordTok{list}\NormalTok{(}\DataTypeTok{train =} \KeywordTok{readRDS}\NormalTok{(}\StringTok{"~/dmc2015/features/feature_files/coupon_basket_stats_train.rds"}\NormalTok{), }
    \DataTypeTok{class =} \KeywordTok{readRDS}\NormalTok{(}\StringTok{"~/dmc2015/features/feature_files/coupon_basket_stats_class.rds"}\NormalTok{))}
\NormalTok{cbXo =}\StringTok{ }\KeywordTok{list}\NormalTok{(}\DataTypeTok{train =} \KeywordTok{readRDS}\NormalTok{(}\StringTok{"~/dmc2015/features/feature_files/coupon_basket_stats_train_byorder.rds"}\NormalTok{), }
    \DataTypeTok{class =} \KeywordTok{readRDS}\NormalTok{(}\StringTok{"~/dmc2015/features/feature_files/coupon_basket_stats_class_byorder.rds"}\NormalTok{))}
\end{Highlighting}
\end{Shaded}

\subsection{Coupon Used}\label{coupon-used}

\begin{Shaded}
\begin{Highlighting}[]
\NormalTok{cu =}\StringTok{ }\KeywordTok{list}\NormalTok{(}\DataTypeTok{train =} \KeywordTok{read.csv}\NormalTok{(}\StringTok{"~/dmc2015/features/feature_files/csv/couponUsed_train.csv"}\NormalTok{), }
    \DataTypeTok{class =} \KeywordTok{read.csv}\NormalTok{(}\StringTok{"~/dmc2015/features/feature_files/csv/couponUsed_class.csv"}\NormalTok{))}
\end{Highlighting}
\end{Shaded}

\subsection{nCoup}\label{ncoup}

\begin{Shaded}
\begin{Highlighting}[]
\NormalTok{nc =}\StringTok{ }\KeywordTok{list}\NormalTok{(}\DataTypeTok{train =} \KeywordTok{read.csv}\NormalTok{(}\StringTok{"~/dmc2015/features/feature_files/csv/nCoupTrain.csv"}\NormalTok{), }
    \DataTypeTok{class =} \KeywordTok{read.csv}\NormalTok{(}\StringTok{"~/dmc2015/features/feature_files/csv/nCoupClass.csv"}\NormalTok{))}
\end{Highlighting}
\end{Shaded}

\section{Putting the pieces together}\label{putting-the-pieces-together}

We don't want any of these variables in our dataset:

\begin{Shaded}
\begin{Highlighting}[]
\NormalTok{bannedvars =}\StringTok{ }\KeywordTok{c}\NormalTok{(}\StringTok{"orderID"}\NormalTok{, }\StringTok{"batchID"}\NormalTok{, }\StringTok{"couponID"}\NormalTok{, }\StringTok{"couponID1"}\NormalTok{, }\StringTok{"couponID2"}\NormalTok{, }\StringTok{"couponID3"}\NormalTok{)}
\end{Highlighting}
\end{Shaded}

We create our X and y using these features:

\begin{Shaded}
\begin{Highlighting}[]
\NormalTok{X =}\StringTok{ }\KeywordTok{data.frame}\NormalTok{(}\DataTypeTok{orderID =} \NormalTok{trn$orderID)}
\NormalTok{y =}\StringTok{ }\NormalTok{trn[, }\KeywordTok{c}\NormalTok{(}\StringTok{"coupon1Used"}\NormalTok{, }\StringTok{"coupon2Used"}\NormalTok{, }\StringTok{"coupon3Used"}\NormalTok{, }\StringTok{"basketValue"}\NormalTok{)]}

\NormalTok{X =}\StringTok{ }\NormalTok{X %>%}\StringTok{ }\KeywordTok{left_join}\NormalTok{(bi$train, }\DataTypeTok{by =} \StringTok{"orderID"}\NormalTok{) %>%}\StringTok{ }\KeywordTok{left_join}\NormalTok{(uv$train, }\DataTypeTok{by =} \StringTok{"orderID"}\NormalTok{) %>%}\StringTok{ }
\StringTok{    }\KeywordTok{left_join}\NormalTok{(cb$train, }\DataTypeTok{by =} \StringTok{"orderID"}\NormalTok{) %>%}\StringTok{ }\KeywordTok{left_join}\NormalTok{(cbXo$train, }\DataTypeTok{by =} \StringTok{"orderID"}\NormalTok{) %>%}\StringTok{ }
\StringTok{    }\KeywordTok{left_join}\NormalTok{(cu$train, }\DataTypeTok{by =} \StringTok{"orderID"}\NormalTok{) %>%}\StringTok{ }\KeywordTok{left_join}\NormalTok{(nc$train, }\DataTypeTok{by =} \StringTok{"orderID"}\NormalTok{) %>%}\StringTok{ }
\StringTok{    }\KeywordTok{left_join}\NormalTok{(trn[, }\KeywordTok{c}\NormalTok{(}\StringTok{"orderID"}\NormalTok{, }\StringTok{"couponsReceived"}\NormalTok{, }\StringTok{"orderTime"}\NormalTok{, }\StringTok{"price1"}\NormalTok{, }\StringTok{"price2"}\NormalTok{, }
        \StringTok{"price3"}\NormalTok{, }\StringTok{"basePrice1"}\NormalTok{, }\StringTok{"basePrice2"}\NormalTok{, }\StringTok{"basePrice3"}\NormalTok{, }\StringTok{"reward1"}\NormalTok{, }\StringTok{"reward2"}\NormalTok{, }
        \StringTok{"reward3"}\NormalTok{, }\StringTok{"premiumProduct1"}\NormalTok{, }\StringTok{"premiumProduct2"}\NormalTok{, }\StringTok{"premiumProduct3"}\NormalTok{)], }
        \DataTypeTok{by =} \StringTok{"orderID"}\NormalTok{)}

\NormalTok{X =}\StringTok{ }\NormalTok{X[, -(}\KeywordTok{which}\NormalTok{(}\KeywordTok{names}\NormalTok{(X) %in%}\StringTok{ }\NormalTok{bannedvars))]}
\end{Highlighting}
\end{Shaded}

and do the same for classification:

\begin{Shaded}
\begin{Highlighting}[]
\NormalTok{X.cls =}\StringTok{ }\KeywordTok{data.frame}\NormalTok{(}\DataTypeTok{orderID =} \NormalTok{cls$orderID)}
\NormalTok{y.cls =}\StringTok{ }\NormalTok{cls[, }\KeywordTok{c}\NormalTok{(}\StringTok{"coupon1Used"}\NormalTok{, }\StringTok{"coupon2Used"}\NormalTok{, }\StringTok{"coupon3Used"}\NormalTok{, }\StringTok{"basketValue"}\NormalTok{)]}

\NormalTok{X.cls =}\StringTok{ }\NormalTok{X.cls %>%}\StringTok{ }\KeywordTok{left_join}\NormalTok{(bi$class, }\DataTypeTok{by =} \StringTok{"orderID"}\NormalTok{) %>%}\StringTok{ }\KeywordTok{left_join}\NormalTok{(uv$class, }
    \DataTypeTok{by =} \StringTok{"orderID"}\NormalTok{) %>%}\StringTok{ }\KeywordTok{left_join}\NormalTok{(cb$class, }\DataTypeTok{by =} \StringTok{"orderID"}\NormalTok{) %>%}\StringTok{ }\KeywordTok{left_join}\NormalTok{(cbXo$class, }
    \DataTypeTok{by =} \StringTok{"orderID"}\NormalTok{) %>%}\StringTok{ }\KeywordTok{left_join}\NormalTok{(cu$class, }\DataTypeTok{by =} \StringTok{"orderID"}\NormalTok{) %>%}\StringTok{ }\KeywordTok{left_join}\NormalTok{(nc$class, }
    \DataTypeTok{by =} \StringTok{"orderID"}\NormalTok{) %>%}\StringTok{ }\KeywordTok{left_join}\NormalTok{(cls[, }\KeywordTok{c}\NormalTok{(}\StringTok{"orderID"}\NormalTok{, }\StringTok{"couponsReceived"}\NormalTok{, }\StringTok{"orderTime"}\NormalTok{, }
    \StringTok{"price1"}\NormalTok{, }\StringTok{"price2"}\NormalTok{, }\StringTok{"price3"}\NormalTok{, }\StringTok{"basePrice1"}\NormalTok{, }\StringTok{"basePrice2"}\NormalTok{, }\StringTok{"basePrice3"}\NormalTok{, }
    \StringTok{"reward1"}\NormalTok{, }\StringTok{"reward2"}\NormalTok{, }\StringTok{"reward3"}\NormalTok{, }\StringTok{"premiumProduct1"}\NormalTok{, }\StringTok{"premiumProduct2"}\NormalTok{, }\StringTok{"premiumProduct3"}\NormalTok{)], }
    \DataTypeTok{by =} \StringTok{"orderID"}\NormalTok{)}

\NormalTok{X.cls =}\StringTok{ }\NormalTok{X.cls[, -(}\KeywordTok{which}\NormalTok{(}\KeywordTok{names}\NormalTok{(X.cls) %in%}\StringTok{ }\NormalTok{bannedvars))]}
\end{Highlighting}
\end{Shaded}

\section{Write the feature matrix}\label{write-the-feature-matrix}

\begin{Shaded}
\begin{Highlighting}[]
\NormalTok{FeatMat3 =}\StringTok{ }\KeywordTok{list}\NormalTok{(}\DataTypeTok{train =} \KeywordTok{list}\NormalTok{(}\DataTypeTok{X =} \NormalTok{X, }\DataTypeTok{y =} \NormalTok{y), }\DataTypeTok{class =} \KeywordTok{list}\NormalTok{(}\DataTypeTok{X =} \NormalTok{X.cls, }\DataTypeTok{y =} \NormalTok{y.cls))}

\KeywordTok{saveRDS}\NormalTok{(FeatMat3, }\DataTypeTok{file =} \StringTok{"~/dmc2015/data/featureMatrix/featMat_v3.0.rds"}\NormalTok{)}
\end{Highlighting}
\end{Shaded}

\end{document}
