% For LaTeX-Box: root = 
%%%%%%%%%%%%%%%%%%%%%%%%%%%%%%%%%%%%%%%%%%%%%%%%%%%%%%%%%%%%%%%%%%%%%%%%%%%%%%%%
%  File Name:
%  Purpose:
%
%  Creation Date: 04-05-2015
%  Last Modified: Mon May  4 18:31:02 2015
%  Created By:
%%%%%%%%%%%%%%%%%%%%%%%%%%%%%%%%%%%%%%%%%%%%%%%%%%%%%%%%%%%%%%%%%%%%%%%%%%%%%%%%

% example yaml
% ---
% title: Homework 2
% titleshort: HW 2
% instructions: Submit all code; show all methods
% author: Ian Mouzon
% authorshort: Mouzon
% contact: imouzon@iastate.edu
% grouplong: Machine Learning
% groupshort: STAT 602
% leader: Dr. Stephen Vardeman
% leadershort: Vardeman
% semester: Spring 2014
% assignment: Problem 1 - 5
% duedate: Monday May 4, 2015
% output:
%   usefulR::hw_template:
% ---

\PassOptionsToPackage{xcolor}{usenames,dvipsnames,svgnames,table}
\documentclass[10pt]{report}
\usepackage[T1]{fontenc}
\usepackage{lmodern}
\usepackage{pdfcolmk}
\usepackage{multirow}
\usepackage{graphicx}
\usepackage{pifont}
\usepackage{amsmath,amsfonts,amsthm,amssymb}
\usepackage{setspace}
\usepackage{Tabbing}
\usepackage{etoolbox}
\usepackage{fancyhdr}
\usepackage{lastpage}
\usepackage{listings}
\usepackage{extramarks}
\usepackage{enumerate}
\usepackage{soul,color}
\usepackage{graphicx,float,wrapfig}
\usepackage{amsmath,amssymb,rotating}
\usepackage{epsfig}
\usepackage{color}
\usepackage{hyperref}
\usepackage{animate}
\usepackage{array}
\usepackage{graphics, color}
\usepackage{graphicx}
\usepackage{epsfig}
\usepackage{setspace}
\usepackage{verbatim}
\usepackage[margin=1.0in]{geometry}
\usepackage{tikz}
\usepackage{mdframed}
\usepackage{clrscode3e}
\usepackage{formalHW}
%\usepackage[Ian Mouzon,]{formatHW}
\usepackage{fancyquote}
\usepackage{fancyenvironments}
\usepackage{mymathmacros}
\usepackage{algorithm}
\usepackage[noend]{algpseudocode}
\usepackage{pgfplots}

%set up fancy page header
%%
\pagestyle{fancy}
   \lhead{\href{mailto:imouzon@iastate.edu}{\nolinkurl{imouzon@iastate.edu}}}
   \lhead{authorshort}
   \chead{STAT 602\ (Vardeman\ Spring 2014): HW 2}  
   \rhead{\firstxmark}
   \lfoot{\lastxmark}
   \cfoot{}
   \rfoot{Page\ \thepage\ of\ \pageref{LastPage}}
   \renewcommand\headrulewidth{0.4pt}
   \renewcommand\footrulewidth{0.4pt}

\begin{document}

%make title:
\makeatletter
\newcommand{\titleheader}{
    \thispagestyle{empty}%
    \begin{center}%
       \renewcommand{\arraystretch}{1.5}%
       \begin{tabular}{c}%
          \Large{STAT 602: Machine Learning}\\
          Homework 2\\
          Spring 2014, Dr.~Stephen Vardeman\\
       \end{tabular}
    \end{center}

    \begin{center}
       \renewcommand{\arraystretch}{1.5}
       \begin{tabular*}{0.65\textwidth}{r@{:\hspace{.3cm}}l}
          \hline
          Name& Ian Mouzon\\
          email& \href{mailto:imouzon@iastate.edu}{\nolinkurl{imouzon@iastate.edu}}\\
          Instructions& Submit all code; show all methods\\
          Assignment& Problem 1 - 5\\
          Due Date&  Monday May 4, 2015\\
          \hline
       \end{tabular*}
    \end{center}
}
\makeatother

I am using the following packages:

\begin{Shaded}
\begin{Highlighting}[]
   \KeywordTok{library}\NormalTok{(ggplot2)}
   \KeywordTok{library}\NormalTok{(lubridate)}
   \KeywordTok{library}\NormalTok{(xtable)}
   \KeywordTok{library}\NormalTok{(foreach)}
   \KeywordTok{library}\NormalTok{(rCharts)}
   \KeywordTok{library}\NormalTok{(magrittr)}
   \KeywordTok{library}\NormalTok{(tidyr)}
   \KeywordTok{library}\NormalTok{(dplyr)}
   \KeywordTok{library}\NormalTok{(reshape2)}
   \KeywordTok{library}\NormalTok{(gtools)}
   \KeywordTok{library}\NormalTok{(sqldf)}
   \KeywordTok{library}\NormalTok{(missForest)}
\end{Highlighting}
\end{Shaded}

and my working directory is set to \verb!dmc2015/ian!.

\section{Reading the Data}\label{reading-the-data}

I am working from the current feature matrix:

\begin{Shaded}
\begin{Highlighting}[]
   \NormalTok{featMat =}\StringTok{ }\KeywordTok{readRDS}\NormalTok{(}\StringTok{"~/dmc2015/data/featureMatrix/featMat_v2.0.rds"}\NormalTok{)}
   \NormalTok{trn =}\StringTok{ }\NormalTok{featMat$train}
   \NormalTok{cls =}\StringTok{ }\NormalTok{featMat$class}

   \CommentTok{#Also reading the melted train and test sets}
   \NormalTok{trn.m =}\StringTok{ }\KeywordTok{read.csv}\NormalTok{(}\StringTok{"~/dmc2015/data/clean_data/melted_train_simple_name.csv"}\NormalTok{)}
   \NormalTok{cls.m =}\StringTok{ }\KeywordTok{read.csv}\NormalTok{(}\StringTok{"~/dmc2015/data/clean_data/melted_test_simple_name.csv"}\NormalTok{)}

   \NormalTok{stack.trn =}\StringTok{ }\NormalTok{trn.m}
   \NormalTok{stack.trn$dsn =}\StringTok{ "trn"}

   \NormalTok{stack.cls =}\StringTok{ }\NormalTok{cls.m}
   \NormalTok{stack.cls$dsn =}\StringTok{ "cls"}

   \NormalTok{stack.m =}\StringTok{ }\KeywordTok{rbind}\NormalTok{(stack.trn,stack.cls)}

   \NormalTok{stack.m$dsn =}\StringTok{ }\KeywordTok{factor}\NormalTok{(stack.m$dsn,}\DataTypeTok{levels=}\KeywordTok{c}\NormalTok{(}\StringTok{'trn'}\NormalTok{,}\StringTok{'cls'}\NormalTok{))}
\end{Highlighting}
\end{Shaded}

In case I need to reference the raw data, I will read that too:

\begin{Shaded}
\begin{Highlighting}[]
   \NormalTok{raw.trn =}\StringTok{ }\KeywordTok{read.csv}\NormalTok{(}\StringTok{"~/dmc2015/data/clean_data/train_simple_name.csv"}\NormalTok{)}
   \NormalTok{raw.cls =}\StringTok{ }\KeywordTok{read.csv}\NormalTok{(}\StringTok{"~/dmc2015/data/clean_data/test_simple_name.csv"}\NormalTok{)}
\end{Highlighting}
\end{Shaded}

\end{document}
