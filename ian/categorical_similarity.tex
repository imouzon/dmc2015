\PassOptionsToPackage{xcolor}{usenames,dvipsnames,svgnames,table}
\documentclass[10pt]{report}
\usepackage[T1]{fontenc}
\usepackage{lmodern}
\usepackage{pdfcolmk}
\usepackage{multirow}
\usepackage{graphicx}
\usepackage{pifont}
\usepackage{amsmath,amsfonts,amsthm,amssymb}
\usepackage{setspace}
\usepackage{Tabbing}
\usepackage{etoolbox}
\usepackage{fancyhdr}
\usepackage{lastpage}
\usepackage{listings}
\usepackage{extramarks}
\usepackage{enumerate}
\usepackage{soul,color}
\usepackage{graphicx,float,wrapfig}
\usepackage{amsmath,amssymb,rotating}
\usepackage{epsfig}
\usepackage{color}
\usepackage{hyperref}
\usepackage{animate}
\usepackage{array}
\usepackage{graphics, color}
\usepackage{graphicx}
\usepackage{epsfig}
\usepackage{setspace}
\usepackage{verbatim}
\usepackage[margin=1.0in]{geometry}
\usepackage{tikz}
\usepackage{mdframed}
\usepackage{clrscode3e}
\usepackage{formalHW}
\usepackage[imouzon,none]{formatHW}
\usepackage{fancyquote}
\usepackage{fancyenvironments}
\usepackage{mymathmacros}
\usepackage{algorithm}
\usepackage[noend]{algpseudocode}
\usepackage{pgfplots}

%set up fancy page header
\pagestyle{fancy}
         \lhead{\href{mailto:imouzon@iastate.edu}{\nolinkurl{imouzon@iastate.edu}}}
            \lhead{imouzon}
      \chead{\href{mailto:DMC@ISU}{\nolinkurl{DMC@ISU}}\ (,\ Spring 2015): CatSims}  
   \rhead{\firstxmark}
   \lfoot{\lastxmark}
   \cfoot{}
   \rfoot{Page\ \thepage\ of\ \pageref{LastPage}}
   \renewcommand\headrulewidth{0.4pt}
   \renewcommand\footrulewidth{0.4pt}

% pandoc syntax highlighting

% header includes

\begin{document}

\thispagestyle{empty}%
\begin{center}%
    \renewcommand{\arraystretch}{1.5}%
    \begin{tabular}{c}%
       \Large{\href{mailto:DMC@ISU}{\nolinkurl{DMC@ISU}}: Iowa State University's 2015 Data Mining Cup Team}\\
       Categorical Similarity Measures\\
       Spring 2015,  \\
    \end{tabular}
\end{center}

\begin{center}
 \renewcommand{\arraystretch}{1.5}
 \begin{tabular*}{0.65\textwidth}{r@{:\hspace{.3cm}}l}
    \hline
     Name& Ian Mouzon\\
     email& \href{mailto:imouzon@iastate.edu}{\nolinkurl{imouzon@iastate.edu}}\\
    
    
     Due Date&  May 8, 2015\\
    \hline
 \end{tabular*}
\end{center}

\section{Categorical Similarity}\label{categorical-similarity}

Oh My Gosh - did you know that you can compare categorical variables to
each other? You can create kind of like a distance on them.

One simple example based on the Jaccard Measure:

Let \(\mathbf{u}\) and \(\mathbf{v}\) be two multidimensional
categorical variables taking values in
\(A = \{a_1, a_2, \ldots, a_n\}\). Then we can think of these as subsets
of the the set \(A\) in which case we can describe a similarity between
the coupons using \[
   J(\mathbf{u}, \mathbf{v}) = \dfrac{| \mathbf{u} \cap \mathbf{v} |}{ \mathbf{u} \cup \mathbf{v}}
\] .

\end{document}
